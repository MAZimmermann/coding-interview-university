%%%%%%%%%%%%%%%%%%%%%%%%%%%%%%%%%%%%%%%%%
%
% Big O Notation
% Notes and Exercises
%
%%%%%%%%%%%%%%%%%%%%%%%%%%%%%%%%%%%%%%%%%

%%%%%%%%%%%%%%%%%%%%%%%%%%%%%%%%%%%%%%%%%
% Short Sectioned Assignment
% LaTeX Template
% Version 1.0 (5/5/12)
%
% This template has been downloaded from: http://www.LaTeXTemplates.com
% Original author: % Frits Wenneker (http://www.howtotex.com)
% License: CC BY-NC-SA 3.0 (http://creativecommons.org/licenses/by-nc-sa/3.0/)
% Modified by Marcus A. Zimmermann MAZimmermann26@gmail.com
%
%%%%%%%%%%%%%%%%%%%%%%%%%%%%%%%%%%%%%%%%%

%----------------------------------------------------------------------------------------
%	PACKAGES AND OTHER DOCUMENT CONFIGURATIONS
%----------------------------------------------------------------------------------------

\documentclass[letterpaper, 10pt,DIV=13]{scrartcl} 

\usepackage[T1]{fontenc} % Use 8-bit encoding that has 256 glyphs
\usepackage[english]{babel} % English language/hyphenation
\usepackage{amsmath,amsfonts,amsthm,xfrac} % Math packages
\usepackage{sectsty} % Allows customizing section commands
\usepackage{graphicx}
\usepackage[lined,linesnumbered,commentsnumbered]{algorithm2e}
\usepackage{listings}
% Added package for listing by letter
\usepackage{enumitem}
%
\usepackage{parskip}
\usepackage{lastpage}

\allsectionsfont{\normalfont\scshape} % Make all section titles in default font and small caps.

\usepackage{fancyhdr} % Custom headers and footers
\pagestyle{fancyplain} % Makes all pages in the document conform to the custom headers and footers

\fancyhead{} % No page header - if you want one, create it in the same way as the footers below
\fancyfoot[L]{} % Empty left footer
\fancyfoot[C]{} % Empty center footer
\fancyfoot[R]{page \thepage\ of \pageref{LastPage}} % Page numbering for right footer

\renewcommand{\headrulewidth}{0pt} % Remove header underlines
\renewcommand{\footrulewidth}{0pt} % Remove footer underlines
\setlength{\headheight}{13.6pt} % Customize the height of the header

\numberwithin{equation}{section} % Number equations within sections (i.e. 1.1, 1.2, 2.1, 2.2 instead of 1, 2, 3, 4)
\numberwithin{figure}{section} % Number figures within sections (i.e. 1.1, 1.2, 2.1, 2.2 instead of 1, 2, 3, 4)
\numberwithin{table}{section} % Number tables within sections (i.e. 1.1, 1.2, 2.1, 2.2 instead of 1, 2, 3, 4)

\setlength\parindent{0pt} % Removes all indentation from paragraphs.

\binoppenalty=3000
\relpenalty=3000

%----------------------------------------------------------------------------------------
%	TITLE SECTION
%----------------------------------------------------------------------------------------

\newcommand{\horrule}[1]{\rule{\linewidth}{#1}} % Create horizontal rule command with 1 argument of height

\title{	
   \normalfont \normalsize
   \horrule{0.5pt} \\[0.25cm] 	% Top horizontal rule
   \huge Big O Notation  \\     	    % Assignment title
   \horrule{0.5pt} \\[0.25cm] 	% Bottom horizontal rule
}

\author{Marcus A. Zimmermann \\ \normalsize MAZimmermann26@gmail.com}

\date{\normalsize\today} 	% Today's date.

\begin{document}
\maketitle % Print the title

%----------------------------------------------------------------------------------------
%   Big O Notation
%----------------------------------------------------------------------------------------
\section*{Big O Notation}

\newcommand{\bslash}{\char`\\}
 
\subsection*{Family of Bachmann-Landau notations}
\setlength{\tabcolsep}{12pt}
\renewcommand*{\arraystretch}{1.2}
\begin{tabular}{lcl}
\hline
Name&Notation\\
\hline
{Big O(micron)}&$\mathcal{O}$ or $O$\\
{Big Omega}&$\Omega$\\
{Big Theta}&$\Theta$\\
{Small O(micron)}&$o$\\
{Small Omega}&$\omega$\\
{On the order of}&$\sim$\\
\hline
\end{tabular}

\subsection*{Formal Definitions}

$T(N)$ =  $O(f(N))$ if there are positive constants \textit{c} and \textit{n}\textsubscript{0} such 
that $T(N)$ $\leq$ $cf(N)$ when $N$ $\geq$ \textit{n}\textsubscript{0} \\

$T(N)$ =  $\Omega(g(N))$ if there are positive constants \textit{c} and \textit{n}\textsubscript{0} such 
that $T(N)$ $\geq$ $cg(N)$ when $N$ $\geq$ \textit{n}\textsubscript{0} \\

$T(N)$ =  $\Theta(h(N))$ if and only if $T(N)$ =  $O(h(N))$ and $T(N)$ =  $\Omega(h(N))$ \\

$T(N)$ =  $o(p(N))$ if $T(N)$ =  $O(p(N))$ and $T(N)$  $\neq$  $\Theta(p(N))$

%----------------------------------------------------------------------------------------
%   end Big O Notation
%---------------------------------------------------------------------------------------

\end{document}